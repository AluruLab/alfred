\documentclass{article}
\usepackage[utf8]{inputenc}
\usepackage[T1]{fontenc}
\usepackage{fixltx2e}
\usepackage{wrapfig}
\usepackage{soul}
\usepackage{textcomp}
\usepackage{amssymb}
\usepackage{hyperref}
\usepackage{graphicx}
\usepackage{longtable}
\usepackage{float}
\usepackage{amsmath,clrscode3e,algorithm,verbatim,tikz}
\usepackage{times}

\author{srirampc}
\date{}
\title{$LCP_k$ Implementation and Experiments}
\begin{document}
\maketitle

\section{Implementation}
\subsection{Outline}

Our algorithm computes $LCP_k$ by a depth-first search of sorted
suffixes allowing upto $k$-mismatches. Output of the algorithm is a
3-dimensional array $KLCP$. $KLCP[1, 1, i]$ records $\max_{j} LCP_k(X_i,
Y_j)$, while $KLCP[1, 2, i]$ records $\arg \max_{j} LCP_k(X_i,
Y_j)$. Similarly, $KLCP[2, \cdot, \cdot]$ records the corresponding
values for $Y$ with respect to $X$.
%$KLCP$ is updated whenever the search frontier reaches a
%depth level of $k$.
$ACS_k(X,Y)$ can be computed using the $KLCP$ arrays using the forumla
given in section (TODO: ref above).

Our algorithm uses Suffix array, ISA, LCP arrays and RMQ tables as the
underlying data-structures. We construct suffix array $SA_R$ for the
string $R$ = $X$\#$Y$\$ using the libdivsufsort library (TODO: cite). We
also construct ISA and LCP arrays $(LCP_R)$, and the RMQ tables
$(RMQ_R)$ based on the implementations from the SDSL library (TODO:
cite). We use the implementations of Kasai Algorithm (TODO: cite) and
Bender-Farach's algorithm (TODO: cite) for construction of LCP arrays
and RMQ tables respectively. Though SDSL library uses bit compression
techniques to reduce the size of the tables and arrays in exchange for
relatively longer time to answer queries, we don't compress these data
structures. We use 32-bit integers for indices and prefix lengths since
they are sufficient for our purposes.

% When the search accomodates an error, the search goes a level
% deeper. When we reach the seach-depth $k$, we have collected a set of
% suffixes from strings $X$ and $Y$ with common prefixes allowing
% $k$-mismatches. Based on this set of suffixes, we update $KLCP$. We
% continue the depth-first search until we run out of suffixes.

% At search depth $1$, we start with all suffixes of $R$, and as we
% go down the levels we consider a subset of suffixes of $R$, constructed
% from the suffixes at the previous level. Search continues until we reach
% a depth of level $k$, and at which point we update $KLCP$ using the
% suffixes collected at the current search frontier.

The following psuedo-code shows the outline of our implementation of the
depth-first search algorithm.
\begin{codebox}
 \Procname{$\proc{Compute-LCPK}()$}
 \li $U \gets \proc{ST-Internal-Nodes}()$
 \li \For $i \gets 1 \To |U|$
 \li \Do $S_1 \gets \proc{ST-Chop-Prefix}(U[i])$
 \li   $\proc{Recursive-Compute-LCPK}(U[i], S_1, 1)$
      \End
 \end{codebox}

The procedure $\proc{Compute-LCPK}$ first calls the function
$\proc{ST-Internal-Nodes}$ to select all the internal nodes of the
suffix tree of $R$. Then, for
each internal node $u$, a call to $\proc{ST-Chop-Prefix}(u)$ chops the
prefixes of length $\proc{String-Depth}(u) + 1$ of all the suffixes
corresponding to $u$. After chopping the prefix, a recursive procedure
$\proc{Recursive-Compute-LCPK}$ is called to explore the search
space further in a depth-first manner.

 \begin{codebox}
 \Procname{$\proc{Recursive-Compute-LCPK}(u, S_{j}, j)$}
 \li \If $j = k$
 \li \Do  $\proc{Update-KLCP}(u, S_{j})$
 \li     \Return
     \End
 \li $U_j \gets \proc{Trie-Internal-Nodes}(u, S_{j})$
 \li \For $i \gets 1 \To |U_{j}|$
 \li \Do $S_{j + 1} \gets \proc{Trie-Chop-Prefix}(U_{j}[i], S_{j})$
 \li   $\proc{Recursive-Compute-LCPK}(U_{j}[i], S_{j + 1}, j + 1)$
      \End
\end{codebox}

The recursive procedure $\proc{Recursive-Compute-LCPK}$ takes as input
an internal node ($u$) of a suffix trie, the corresponding set of sorted
suffixes ($S_j$) and the search depth ($j$). When the search depth
reaches $k$, recursion ends and $KLCP$ arrays are updated. Otherwise,
the procedure selects the internal nodes of the suffix trie induced by
$S_j$, chops the common prefix of the suffixes corresponding to each
internal node in the trie, and recursively calls itself while
incrementing the search depth.

% We always consider a set of suffixes under the context of a specific
% internal node $u$. $u$ is either an internal node in a suffix tree (at
% search depth of $0$) or an internal node of a suffix trie (at search
% depths $1, \ldots, k-1$). We construct suffixes at search level $j$ by
% chopping prefix of the suffixes under at search level $j-1$. Chopping
% one character after matching the common prefix of the suffixes allows
% collecting the suffixes with one mismatch in their common prefixes.
\subsection{Representation}

The procedures shown above represents sets of internal nodes and
suffixes as arrays of tuples. We represent an interal node $u$ (both for
the suffix tree and the suffix tries) as a tuple $(bp, ep, d, \delta
)$. Here, $bp$ and $ep$ are the start and end indices pointing to a
sorted array of suffixes. At search level $0$, this array is $SA_R$, and
$bp$ and $ep$ are indices to $SA_R$. At search levels $j =
1,\ldots,k-1$, they are a subset of suffixes selected from $R$. $d$ is
the string depth of the internal node in this trie, and $\delta$ the sum
of the chopped lengths at the previous levels $1, \ldots, j-1$. At level
$0$, $\delta = 0$.

Suffixes are considered under the context of an internal node $u$. At
search level $0$, suffixes are just indices of $R$. At search levels $j
= 1,\ldots,k-1$, we represent the suffixes of the trie with the tuple
$(c, c', str)$, where $c$ is the position in the string $R$, $c'$ is the
position in $SA_R$ of suffix $c$ after chopping the first
$\proc{String-Depth}(u) + 1$ characters. $str$ field indicates whether
the suffix is from either $X$ or $Y$. In this paper, we use function
notation to access a specific element of the tuple. For example, to
access field $c$ of the suffix $x$ represented by tuple $(i, j, 2)$,
we write $c(r) = i$.

Since $\# <$ \$ $< \{A,\ldots, Z\}$, $SA_R[1]$ and $SA_R[2]$ are $\#Y$
and \$ -- the end of first and second strings -- respectively.
As we chop off suffixes and reach towards the end of the strings, we
might end up in either at the first or second entry of the suffix
array. When we proceed to the next search level, we ignore these suffixes.

\subsection{Procedures}

As noted above, $\proc{ST-Internal-Nodes}$ selects the internal nodes of
the suffix tree of $R$. In this procedure , there is no explicit
construction of the suffix tree. For our purposes, we require only the
left and right bounds of the sub-tree corresponding to all internal
nodes, which can be computed using only $SA_R$ and $LCP_R$ as
follows. $LCP_R[i+1]$ records the string depth of the least common
ancestor $u$ of two consecutive pair of suffixes $(i, i+1)$ in
$SA_R$. The left and right bounds of $u$ can be computed by enclosing
the pair $(i, i+1)$ with all the suffixes having the same least common
ancestor $u$. By collecting suffixes of the least common ancestor of
every pair $(i, i+1), i = 3, \ldots, |SA_R|$, the procedure
$\proc{ST-Internal-Nodes}$ computes the left and right bounds for all
the internal nodes in the suffix tree of $R$.

$\proc{ST-Chop-Prefix}$ chops the common prefixes of the suffixes and
construct the set of chopped suffix tuples. It also sorts the chopped
suffix tuples with $c'$ as the key.

$\proc{Trie-Internal-Nodes}$ selects left and right bounds of the
suffixes corresponding to the internal nodes of the trie. Similar to the
suffix tree right and left bounds can be computed by enclosing a pair of
suffixes, but instead of $LCP_R$ arrays, we use the $RMQ_R$ tables to
retrieve the length of least common prefixes of two suffixes in a trie.
Similar to the suffix tree counterpart, $\proc{Trie-Chop-Prefix}$ chops
common prefix of a set of suffixes allowing one mismatch and sorts the
suffixes by the $c'$.

The $\proc{Update-KLCP}$ procedure, as shown below, makes two passes one
from left to right and another from right to left.

\begin{codebox}
\Procname{$\proc{Update-KLCP}(u, S_j)$}
\li $\proc{Update-KLCP-Left-To-Right}(u, S_j)$
\li $\proc{Update-KLCP-Right-To-Left}(u, S_j)$
\End
\end{codebox}
Except for the direction of scan, both left-to-right and right-to-left
passes are similiar in that they scan through the array of suffixes to
identify the pairs of suffixes $(p, q), p < q$ such that one each from
$X$ and $Y$ to update the corresponding $KLCP$ entries.
$\proc{Update-KLCP-Left-To-Right}(u, S_j)$ is as shown below.
\begin{codebox}
\Procname{$\proc{Update-KLCP-Left-To-Right}(u, S_j)$}
\li $p \gets 1$; $q \gets 2$; $L_H[1] = 0$; $L_H[2] = 1 + |X|$
    \label{up:init0}
\li \While $q < |S_j|$ \kw{and} $src(S_j[p]) = src(S_j[q])$
\li \Do $p \gets q$; $q \gets q + 1$ \label{up:init1}
\End
\li \While $q < |S_j|$
\li \Do $r \gets c(S_j[p]) - L_H[src(S_j[p])] - \delta(u)$  %\Comment Starting Position
\label{up:score0}
\li     $t \gets c(S_j[q]) - L_H[src(S_j[q])] - \delta(u)$
\li  $l_{p,q} \gets \delta(u) + d(u) + RMQ_R(c'(S_j[p]), c'(S_j[q])) + 1$
\li  \If $l_{p,q} > KLCP[src(S_j[q])][1][t]$
\li    \Do $KLCP[src(S_j[q])][1][t]  \gets l_{p,q}$
\li        $KLCP[src(S_j[q])][2][t] \gets r$
\End \label{up:score1}
\li \If $src(S_j[p]) == src(S_j[q + 1])$ \label{up:move0}
\li  \Then $p = q$ \End
\li   $q \gets q + 1$ \label{up:move1}
\End
\end{codebox}
Lines \ref{up:init0} -- \ref{up:init1} initializes pointers $p$ and $q$
to the first consecutive suffixes in $S_j$ such that $S_j[p]$ and
$S_j[q]$ originate from two different strings. Lines \ref{up:score0} --
\ref{up:score1} evaluate the length of longest common prefix with
$k$-mismatches between suffixes $S_j[p]$ and $S_j[q]$. If the evaluted
prefix length is longer than current length recored by $KLCP$ for
$src(S_j[q])$, it updates the corresponding $KLCP$
entries. Lines \label{up:move0} -- \label{up:move1} updates the pointers
$p$ and $q$ to the next pair of suffixes that from originate from two
different strings.

\end{document}